% Options for packages loaded elsewhere
\PassOptionsToPackage{unicode}{hyperref}
\PassOptionsToPackage{hyphens}{url}
%
\documentclass[
]{article}
\usepackage{amsmath,amssymb}
\usepackage{lmodern}
\usepackage{ifxetex,ifluatex}
\ifnum 0\ifxetex 1\fi\ifluatex 1\fi=0 % if pdftex
  \usepackage[T1]{fontenc}
  \usepackage[utf8]{inputenc}
  \usepackage{textcomp} % provide euro and other symbols
\else % if luatex or xetex
  \usepackage{unicode-math}
  \defaultfontfeatures{Scale=MatchLowercase}
  \defaultfontfeatures[\rmfamily]{Ligatures=TeX,Scale=1}
\fi
% Use upquote if available, for straight quotes in verbatim environments
\IfFileExists{upquote.sty}{\usepackage{upquote}}{}
\IfFileExists{microtype.sty}{% use microtype if available
  \usepackage[]{microtype}
  \UseMicrotypeSet[protrusion]{basicmath} % disable protrusion for tt fonts
}{}
\makeatletter
\@ifundefined{KOMAClassName}{% if non-KOMA class
  \IfFileExists{parskip.sty}{%
    \usepackage{parskip}
  }{% else
    \setlength{\parindent}{0pt}
    \setlength{\parskip}{6pt plus 2pt minus 1pt}}
}{% if KOMA class
  \KOMAoptions{parskip=half}}
\makeatother
\usepackage{xcolor}
\IfFileExists{xurl.sty}{\usepackage{xurl}}{} % add URL line breaks if available
\IfFileExists{bookmark.sty}{\usepackage{bookmark}}{\usepackage{hyperref}}
\hypersetup{
  pdftitle={Exam2Pomeroy},
  pdfauthor={Pomeroy},
  hidelinks,
  pdfcreator={LaTeX via pandoc}}
\urlstyle{same} % disable monospaced font for URLs
\usepackage[margin=1in]{geometry}
\usepackage{color}
\usepackage{fancyvrb}
\newcommand{\VerbBar}{|}
\newcommand{\VERB}{\Verb[commandchars=\\\{\}]}
\DefineVerbatimEnvironment{Highlighting}{Verbatim}{commandchars=\\\{\}}
% Add ',fontsize=\small' for more characters per line
\usepackage{framed}
\definecolor{shadecolor}{RGB}{248,248,248}
\newenvironment{Shaded}{\begin{snugshade}}{\end{snugshade}}
\newcommand{\AlertTok}[1]{\textcolor[rgb]{0.94,0.16,0.16}{#1}}
\newcommand{\AnnotationTok}[1]{\textcolor[rgb]{0.56,0.35,0.01}{\textbf{\textit{#1}}}}
\newcommand{\AttributeTok}[1]{\textcolor[rgb]{0.77,0.63,0.00}{#1}}
\newcommand{\BaseNTok}[1]{\textcolor[rgb]{0.00,0.00,0.81}{#1}}
\newcommand{\BuiltInTok}[1]{#1}
\newcommand{\CharTok}[1]{\textcolor[rgb]{0.31,0.60,0.02}{#1}}
\newcommand{\CommentTok}[1]{\textcolor[rgb]{0.56,0.35,0.01}{\textit{#1}}}
\newcommand{\CommentVarTok}[1]{\textcolor[rgb]{0.56,0.35,0.01}{\textbf{\textit{#1}}}}
\newcommand{\ConstantTok}[1]{\textcolor[rgb]{0.00,0.00,0.00}{#1}}
\newcommand{\ControlFlowTok}[1]{\textcolor[rgb]{0.13,0.29,0.53}{\textbf{#1}}}
\newcommand{\DataTypeTok}[1]{\textcolor[rgb]{0.13,0.29,0.53}{#1}}
\newcommand{\DecValTok}[1]{\textcolor[rgb]{0.00,0.00,0.81}{#1}}
\newcommand{\DocumentationTok}[1]{\textcolor[rgb]{0.56,0.35,0.01}{\textbf{\textit{#1}}}}
\newcommand{\ErrorTok}[1]{\textcolor[rgb]{0.64,0.00,0.00}{\textbf{#1}}}
\newcommand{\ExtensionTok}[1]{#1}
\newcommand{\FloatTok}[1]{\textcolor[rgb]{0.00,0.00,0.81}{#1}}
\newcommand{\FunctionTok}[1]{\textcolor[rgb]{0.00,0.00,0.00}{#1}}
\newcommand{\ImportTok}[1]{#1}
\newcommand{\InformationTok}[1]{\textcolor[rgb]{0.56,0.35,0.01}{\textbf{\textit{#1}}}}
\newcommand{\KeywordTok}[1]{\textcolor[rgb]{0.13,0.29,0.53}{\textbf{#1}}}
\newcommand{\NormalTok}[1]{#1}
\newcommand{\OperatorTok}[1]{\textcolor[rgb]{0.81,0.36,0.00}{\textbf{#1}}}
\newcommand{\OtherTok}[1]{\textcolor[rgb]{0.56,0.35,0.01}{#1}}
\newcommand{\PreprocessorTok}[1]{\textcolor[rgb]{0.56,0.35,0.01}{\textit{#1}}}
\newcommand{\RegionMarkerTok}[1]{#1}
\newcommand{\SpecialCharTok}[1]{\textcolor[rgb]{0.00,0.00,0.00}{#1}}
\newcommand{\SpecialStringTok}[1]{\textcolor[rgb]{0.31,0.60,0.02}{#1}}
\newcommand{\StringTok}[1]{\textcolor[rgb]{0.31,0.60,0.02}{#1}}
\newcommand{\VariableTok}[1]{\textcolor[rgb]{0.00,0.00,0.00}{#1}}
\newcommand{\VerbatimStringTok}[1]{\textcolor[rgb]{0.31,0.60,0.02}{#1}}
\newcommand{\WarningTok}[1]{\textcolor[rgb]{0.56,0.35,0.01}{\textbf{\textit{#1}}}}
\usepackage{graphicx}
\makeatletter
\def\maxwidth{\ifdim\Gin@nat@width>\linewidth\linewidth\else\Gin@nat@width\fi}
\def\maxheight{\ifdim\Gin@nat@height>\textheight\textheight\else\Gin@nat@height\fi}
\makeatother
% Scale images if necessary, so that they will not overflow the page
% margins by default, and it is still possible to overwrite the defaults
% using explicit options in \includegraphics[width, height, ...]{}
\setkeys{Gin}{width=\maxwidth,height=\maxheight,keepaspectratio}
% Set default figure placement to htbp
\makeatletter
\def\fps@figure{htbp}
\makeatother
\setlength{\emergencystretch}{3em} % prevent overfull lines
\providecommand{\tightlist}{%
  \setlength{\itemsep}{0pt}\setlength{\parskip}{0pt}}
\setcounter{secnumdepth}{-\maxdimen} % remove section numbering
\ifluatex
  \usepackage{selnolig}  % disable illegal ligatures
\fi

\title{Exam2Pomeroy}
\author{Pomeroy}
\date{6/28/2021}

\begin{document}
\maketitle

\hypertarget{r-markdown}{%
\subsection{R Markdown}\label{r-markdown}}

This is an R Markdown document. Markdown is a simple formatting syntax
for authoring HTML, PDF, and MS Word documents. For more details on
using R Markdown see \url{http://rmarkdown.rstudio.com}.

When you click the \textbf{Knit} button a document will be generated
that includes both content as well as the output of any embedded R code
chunks within the document. You can embed an R code chunk like this:

\begin{Shaded}
\begin{Highlighting}[]
\FunctionTok{summary}\NormalTok{(cars)}
\end{Highlighting}
\end{Shaded}

\begin{verbatim}
##      speed           dist       
##  Min.   : 4.0   Min.   :  2.00  
##  1st Qu.:12.0   1st Qu.: 26.00  
##  Median :15.0   Median : 36.00  
##  Mean   :15.4   Mean   : 42.98  
##  3rd Qu.:19.0   3rd Qu.: 56.00  
##  Max.   :25.0   Max.   :120.00
\end{verbatim}

\hypertarget{including-plots}{%
\subsection{Including Plots}\label{including-plots}}

You can also embed plots, for example:

\includegraphics{Pomeroy-exam-2_files/figure-latex/pressure-1.pdf}

Note that the \texttt{echo\ =\ FALSE} parameter was added to the code
chunk to prevent printing of the R code that generated the plot.

\hypertarget{clear-environment}{%
\section{clear environment}\label{clear-environment}}

rm(list = ls(all(TRUE)))

rm(list = ls())

\hypertarget{load-dataset}{%
\section{load dataset}\label{load-dataset}}

college\_scorecard \textless- X2021\_exam2\_data

\hypertarget{remove-x2021-df}{%
\section{remove x2021 df}\label{remove-x2021-df}}

rm(X2021\_exam2\_data)

\hypertarget{load-tidyverse}{%
\section{load tidyverse}\label{load-tidyverse}}

library(tidyverse)

\hypertarget{summary-college_scorecard}{%
\section{summary college\_scorecard}\label{summary-college_scorecard}}

summary(college\_scorecard)

\hypertarget{summary-class}{%
\section{summary class}\label{summary-class}}

summary(college\_scorecard\$class)

\hypertarget{create-small_scorecard}{%
\section{create small\_scorecard}\label{create-small_scorecard}}

small\_scorecard \textless- subset(college\_scorecard, select =
c(``year'', ``state\_abbr''))

\hypertarget{dataset-didnt-load-all-the-sheets-try-this-again}{%
\section{dataset didn't load all the sheets :( try this
again}\label{dataset-didnt-load-all-the-sheets-try-this-again}}

\hypertarget{load-dataset-1}{%
\section{load dataset}\label{load-dataset-1}}

college\_scorecard \textless- X2021\_exam2\_data

\hypertarget{make-small_scorecard}{%
\section{make small\_scorecard}\label{make-small_scorecard}}

small\_scorecard \textless- subset(college\_scorecard, select =
c(``year'', ``state\_abbr''))

\hypertarget{sorry-i-am-completely-blanking-on-how-to-do-for-loops-right-now-will-try-to-come-back-to-part-1}{%
\section{sorry I am completely blanking on how to do for loops right now
:/ will try to come back to part
1}\label{sorry-i-am-completely-blanking-on-how-to-do-for-loops-right-now-will-try-to-come-back-to-part-1}}

\hypertarget{load-avocados}{%
\section{load avocados}\label{load-avocados}}

avocados \textless- X2021\_exam2\_data

\hypertarget{year-variable}{%
\section{year variable}\label{year-variable}}

avocados \%\textgreater\% year \textless- select(year)

\hypertarget{hmm-nothing-is-really-working-try-reloading-everything}{%
\section{hmm nothing is really working, try reloading
everything}\label{hmm-nothing-is-really-working-try-reloading-everything}}

install.packages(``tidyverse'') install.packages(``ggplot2'')
install.packages(``lubridate'')

library(tidyverse) library(ggplot2) library(lubridate)

\hypertarget{try-making-year-variable-for-avocados}{%
\section{try making year variable for
avocados}\label{try-making-year-variable-for-avocados}}

avocados \%\textgreater\% year \textless- select(avocados\$date(year)
\%\textgreater\% mutate(year - make\_datetime(year))

\#can't remember how to fix the select vs ``function'' error

\hypertarget{deflate-the-average_price-variable}{%
\section{deflate the average\_price
variable}\label{deflate-the-average_price-variable}}

\hypertarget{load-wdi-deflation-info}{%
\section{load WDI deflation info}\label{load-wdi-deflation-info}}

install.packages(``WDI'') library(WDI)

\hypertarget{deflate-average-price-variable}{%
\section{deflate average price
variable}\label{deflate-average-price-variable}}

deflated\_price\_2015 \textless- WDI(indicator =
``API\_NY.GDP.DEFL.KD.ZG\_DS2\_en\_excel\_v2\_2446026'')

\hypertarget{hmm-dont-think-i-found-the-correct-file-for-deflation}{%
\section{hmm don't think I found the correct file for
deflation}\label{hmm-dont-think-i-found-the-correct-file-for-deflation}}

\hypertarget{httpsdata.worldbank.orgindicatorny.gdp.defl.zs-try-importing-this-one}{%
\section{\texorpdfstring{\url{https://data.worldbank.org/indicator/NY.GDP.DEFL.ZS}
try importing this
one}{https://data.worldbank.org/indicator/NY.GDP.DEFL.ZS try importing this one}}\label{httpsdata.worldbank.orgindicatorny.gdp.defl.zs-try-importing-this-one}}

\hypertarget{create-deflated-price-variable-2015}{%
\section{create deflated price variable
2015}\label{create-deflated-price-variable-2015}}

deflated\_price\_2015 \textless- WDI(country = ``all'', indicator =
c(``API\_NY\_GDP\_DEFL\_ZS''), start = 2015, end = 2015, extra = FALSE,
cache = NULL)

\hypertarget{not-really-sure-whats-going-on-with-trying-to-pull-the-data-hmm}{%
\section{not really sure what's going on with trying to pull the data
hmm}\label{not-really-sure-whats-going-on-with-trying-to-pull-the-data-hmm}}

\hypertarget{indicator-is-1.041461871}{%
\section{2015 indicator is 1.041461871}\label{indicator-is-1.041461871}}

deflator\_2015 \textless- 1.041461871

\hypertarget{indicator-is-1.035268879}{%
\section{2016 indicator is 1.035268879}\label{indicator-is-1.035268879}}

deflator\_2016 \textless- 1.035268879

\hypertarget{indicator-is-1.88368859}{%
\section{2017 indicator is 1.88368859}\label{indicator-is-1.88368859}}

deflator\_2017 \textless- 1.88368859

\hypertarget{indicator-is-2.436001896}{%
\section{2018 indicator is 2.436001896}\label{indicator-is-2.436001896}}

deflator\_2018 \textless- 2.436001896

deflator deflated\_price\_2015 \textless-

\#skipping this, collapsing df =

head(avocados)

collapsed\_avocados \textless-

avocados \%\textgreater\%

group\_by(c(deflated\_price\_2015, deflated\_price\_2016,
deflated\_price\_2017, deflated\_price\_2018))

\hypertarget{then-i-would-check-by-headcollapsed_avocados-if-i-had-done-the-rest-of-this-problem-right}{%
\section{then I would check by head(collapsed\_avocados if I had done
the rest of this problem
right)}\label{then-i-would-check-by-headcollapsed_avocados-if-i-had-done-the-rest-of-this-problem-right}}

head(collapsed\_avocados)

\hypertarget{reshaping-the-guacamole-wider}{%
\section{reshaping the guacamole
wider}\label{reshaping-the-guacamole-wider}}

collapsed\_avocados \%\textgreater\%

wide avocados \textless- pivot\_wider(id\_cols = c(``date'',
``average\_price'') names\_from = ``date'', values\_from =
``average\_price'')

\hypertarget{im-running-out-of-time-whoops}{%
\section{I'm running out of time :/
whoops}\label{im-running-out-of-time-whoops}}

\hypertarget{training-dataset-qs}{%
\section{training dataset q's}\label{training-dataset-qs}}

training \textless- X2021\_exam2\_data

\hypertarget{reshape-the-df-long}{%
\section{reshape the df long}\label{reshape-the-df-long}}

pivot\_longer(cols = starts\_with(""))

\end{document}
